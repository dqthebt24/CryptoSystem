Theo tham khảo tài liệu trên Iternet \footnote{https://t.ly/zGU9}\footnote{https://t.ly/LXCr} thì trong RSA thường chọn $e$ là các số nguyên tố nhỏ trong các số Fermat 
\footnote{https://t.ly/nieN} ($e \in \{3,5,17,257,65537\}$) 
rồi từ $e$ ta chọn $d$ sao cho $ed = 1\ mod\ \phi$.
Tuy nhiên, như trong giáo trình môn học, để an toàn hơn không nên chọn trước số $e$ nhỏ rồi chọn $d$ vì $d$ có khả năng sẽ nhỏ, nên chọn các số $d \geq \sqrt[4]{n}$.

Ở đây đồ án thực hiện chọn $e, d$ bằng cách chọn $e$ là các số nguyên tố $\geq 63357$, chọn $d$ tương ứng với mỗi $e$ được chọn sau đó kiểm tra điều kiện $d \geq \sqrt[4]{n}$ 
nếu $d$ thỏa thì chọn cặp $e, d$. \textit{Thuật toán \ref{alg:gen-ed}} minh họa quá trình tạo $e$, $d$

Thuật toán tìm $d$ khi biết $e, \phi$ được dùng là thuật toán Bezout nhị phân tham khảo từ tài liệu \cite{giaotrinh} (Phần thuật giải 1.49, trang 26).

\begin{algorithm}
\caption{Thuật toán tạo key $e$, $d$}\label{alg:gen-ed}
\hspace*{\algorithmicindent} \textbf{Input:} $p, q$: Hai số nguyên tố lớn\\
\hspace*{\algorithmicindent} \textbf{Output:} $e$, $d$ sao cho $ed = \ 1\ mod\ \phi\ ,\phi\ =\ (p\ -\ 1)(q\ -\ 1)$ 
\begin{algorithmic}[1]
\Procedure{GenKeys}{$p, q$}
\State $n \gets p*q$
\State $\phi \gets (p\ -\ 1)(q\ -\ 1)$
\State $t \gets 10922$ \Comment $6t + 5 = 65537$
\While {true}
\State $e \gets 6t + 1$
\If {\textit{ISPRIME($e$)}}
\State $d \gets BinaryBezout(e, \phi)$
\If {$d \geq \sqrt[4]{n}$}
\State \textbf{break}
\EndIf
\EndIf
\State $e \gets 6t+5$
\If {\textit{ISPRIME($e$)}}
\State $d \gets BinaryBezout(e, \phi)$
\If {$d \geq \sqrt[4]{n}$}
\State \textbf{break}
\EndIf
\EndIf
\State $t \gets t + 1$
\EndWhile
\State \textbf{Return } $e$, $d$
\EndProcedure
\end{algorithmic}
\end{algorithm}