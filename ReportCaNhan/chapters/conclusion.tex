Đồ án đã thực hiện cài đặt lại và thực nghiệm thuật toán mã hóa RSA bằng 2 cách, nhìn chung cả hai cách đều không dùng thư viện số nguyên lớn mà tự cài đặt lại các phép toán. 
Cách 2 dùng cách lưu trữ số nguyên dựa vào thư viện \textit{boost::multiprecision} cho thấy thời gian chạy tốt hơn, thời gian viết chương trình nhanh hơn trong 
khi tự cài đặt lại \textit{class BigInt} tốn thời gian code (vì cần phải unittest để kiểm tra tính đúng đắn của code) mà thời gian chạy cũng không được nhanh, hơn 
nữa việc cài đặt này là không cần thiết vì \textit{``reinvent the wheel``} không phải là cách hay.

Nhược điểm chủ yếu của việc viết lại lớp số nguyên lớn là yêu cầu phải xử lý số nguyên có dấu (vì các thuật toán như Bezout nhị phân hay Euclid mở rộng đều cần thao tác với số có dấu) tuy nhiên việc xử lý số có dấu tốn khá nhiều thời gian implement. Hơn nữa, với cách làm hiện tại thì việc chậm là hiển nhiên vì nếu số 1024 bits tức là 1024 \textit{char} trong con trỏ \textit{char*} $\rightarrow$ cần xử lý $8*1024 = 8192$ bits! Vậy nên để cải thiện tốc độ cần tối ưu thêm ở mức lưu trữ các bits số mà hiện tại \textit{boost::multiprecision} đã thực hiện điều đó.

