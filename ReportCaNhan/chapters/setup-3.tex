Cũng như việc cài đặt phép tính \%, việc cài đặt phép tính MulMod dùng kiểu dữ liệu \textbf{number\_t} đã định nghĩa không có gì đặc biệt ngoài việc đảm bảo 
không bị tràn số, tuy nhiên với kiểu dữ liệu 4096 bits thì việc tràn số trong phép nhân với hai số 1024 bits không thể xảy ra vì phép nhân số $n$ bits và số $t$ bits 
cho ra kết quả nhiều nhất là $n + t + 2$ bits (\textit{\cite{Menezes} Chương 14 trang 595}). Thực hiện phép MulMod với kiểu dữ liệu \textbf{number\_t} là thực hiện phép nhân 
trước rồi phép modulo sau.

Còn với cách 1, cài đặt thuật toán MulMod cho kiểu số nguyên lớn tự định nghĩa được thực hiện theo \textit{Thuật toán \ref{alg:mul-mod}} (tham khảo giáo trình).

\begin{algorithm}
\caption{Tính MulMod}\label{alg:mul-mod}
\hspace*{\algorithmicindent} \textbf{Input:} BigInt x, BigInt y, BigInt n\\
\hspace*{\algorithmicindent} \textbf{Output:} x*y\ \%\ n
\begin{algorithmic}[1]
\Procedure{MulMod}{$x,y,n$}
\State $p \gets 0$
\If {$y_0 = 1$} \Comment Bit cuối của y là 1
\State $p \gets x$
\EndIf
\For {$i \gets 1$, m}\Comment m là số bit của y
\State $x \gets 2*x\ \%\ n$ \Comment x L-Shift 1 bit rồi \% n
\If {$y_i = 1$}
\State $p \gets (p+x)\ \%\ n$
\EndIf
\EndFor
\State \textbf{Return } $p$
\EndProcedure
\end{algorithmic}
\end{algorithm}