Định lí số dư trung hoa có nội dung như sau: Với $n_i \in \mathcal{P}, n_i \neq n_j\ \forall i \neq j$ và $a_i \in \mathbb{N}$. Hệ phương trình sau có nghiệm 
duy nhất trong $\mathbb{Z}_{n_1n_2\hdots n_k}$
\begin{equation}\label{sequations:abs-crt}
\begin{cases}
x &= a_1\ mod\ n_1 \\
x &= a_2\ mod\ n_2 \\
& \vdots \\
x &= a_k\ mod\ n_k
\end{cases}
\end{equation}

Chứng minh của Quisquater \& Couvreur trong \cite{Quisquater1982} thể hiện việc áp dụng CRT trong RSA như sau: Với $p, q \in \mathcal{P}, p \neq q$, $\phi\ =\ (p -1)(q-1)$, 
$e,d \in \mathbb{N}, gcd(e,\phi)\ =\ 1, ed = 1\ mod\ \phi$, $n = pq$, nếu $x = c^d\ mod \ n$ thì $x$ cũng là nghiệm của hệ phương trình:

\begin{equation}\label{sequations:crt-rsa}
\begin{cases}
x &= c^{d_1}\ mod\ p \\
x &= c^{d_2}\ mod\ q
\end{cases}
\end{equation}
\begin{center}
với $d_1\ = d\ mod\ (p - 1), d_2\ =\ d\ mod\ (q - 1) $
\end{center}

Vận dụng \textit{(\ref{sequations:crt-rsa})}, quá trình \textit{decrypt} được cải thiện rõ rệt vì thay vì tính $c^d$ và modulo $n$ thì kết quả có thể tính 
được bằng các phép tính tương tự nhưng với số bit của $p, q$ nhỏ hơn nhiều so với số bit của $d, n$.

Cài đặt \textit{decrypt nhanh} là cài đặt giải hệ phương trình CRT tổng quát \textit{(\ref{sequations:abs-crt})} từ đó giải \textit{(\ref{sequations:crt-rsa})}. 

Cho trước hai số nguyên $a,b$ định lý Bezout phát biểu có thể tìm được $x$ và $y$ thỏa $ax + by = gcd(a,b)$. Trong \textit{hệ phương trình 
(\ref{sequations:abs-crt})} ta thấy với $n=n_1n_2\hdots n_k, n_i \in \mathcal{P}$ thì $gcd(n_i,\frac{n}{n_i}) = 1$. Như vậy bằng thuật toán Euclid mở rộng (hoặc \textit{Bezout nhị phân} \cite{giaotrinh} như đã nêu ở phần trên) có thể tìm được 
$r_i$ và $s_i$ sao cho $n_i r_i + \frac{n}{n_i} s_i = 1$, từ đó ta có thể tìm được nghiệm duy nhất của \textit{(\ref{sequations:abs-crt})} là $x = \sum_{i=1}^{k}a_is_i\frac{n}{n_i}$. 
\textit{Thuật toán \ref{alg:solve-crt}} dùng để giải phương trình CRT tổng quát.

Ngoài ra, nếu tính và lưu trước giá trị $s_1\frac{n}{p}$ và $s_2\frac{n}{q}$ cùng với việc tính và lưu $e, d$ trong quá trình tạo keys của RSA thì khi \textit{decrypt} sẽ không phải 
chạy lại thuật toán tìm $s_i$ do đó thời gian \textit{decrypt} có thể được rút ngắn hơn nữa.

\begin{algorithm}[H]
\caption{Giải hệ phương trình CRT}\label{alg:solve-crt}
\hspace*{\algorithmicindent} \textbf{Input:} listA, listN: Danh sách $a_i$ và $n_i$\\
\hspace*{\algorithmicindent} \textbf{Output:} x: Nghiệm của hệ phương trình CRT
\begin{algorithmic}[1]
\Procedure{SolveCrt}{listA, listN}
\State $len \gets \textit{sizeof(listN)}$
\State $n \gets 0$
\State $sum \gets 0$
\For {i \textbf{in} listN}
\State $n \gets n*i$
\EndFor
\For {$i \gets 1$, len}
\State $n_i \gets listN[i]$
\State $p \gets n/n_i$
\State $a_i \gets listA[i]$
\State $r,s \gets$ BinaryBezout($n_i$, p)
\State $sum \gets a_i*s*p$
\EndFor
\State \textbf{Return } $sum\ \%\ n$
\EndProcedure
\end{algorithmic}
\end{algorithm}