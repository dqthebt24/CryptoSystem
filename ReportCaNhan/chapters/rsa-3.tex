Theo định lí CRT, với $p, q \in \mathcal{P}, q \neq q$ nếu $x = c^d\ mod \ n$, $n = pq$ thì $x$ cũng là nghiệm của hệ phương trình:
\begin{equation}
\begin{cases}
x &= c^{d_1}\ mod\ p \\
x &= c^{d_2}\ mod\ q
\end{cases}
\end{equation}
\begin{center}
với $d_1\ = d\ mod\ (p - 1), d_2\ =\ d\ mod\ (q - 1) $
\end{center}

Quá trình \textit{decrypt} cũng được cải thiện rõ rệt nhờ việc dùng định lí số dư Trung Hoa (CRT) vì thay vì tính $c^d$ và modulo $n$ thì kết quả có thể tính 
được bằng các phép tính tương tự nhưng với số bit của $d_1, d_2, p, q$ nhỏ hơn nhiều so với số bit của $d, n$.

Cài đặt \textit{decrypt nhanh} là cài đặt giải hệ phương trình CRT tổng quát:
\begin{equation}\label{sequations:abs-crt}
\begin{cases}
x &= a_1\ mod\ n_1 \\
x &= a_2\ mod\ n_2 \\
& \vdots \\
x &= a_k\ mod\ n_k
\end{cases}
\end{equation}
\begin{center}
với $n_i \in \mathcal{P}, n_i \neq n_j\ \forall i \neq j$
\end{center}

Cho trước hai số nguyên $a,b$, bằng thuật toán Euclid mở rộng ta có thể tìm được $x$ và $y$ thõa $ax + by = gcd(a,b)$. Trong \textit{hệ phương trình (\ref{sequations:abs-crt})} 
ta thấy với $n=n_1n_2\hdots n_k$ thì $gcd(n_i,\frac{n}{n_i}) = 1$. Như vậy ta có thể tìm được $r_i$ và $s_i$ sao cho $n_i r_i + \frac{n}{n_i} s_i = 1$, 
từ đó ta có thể tìm được nghiệm duy nhất của \textit{(\ref{sequations:abs-crt})} là $x = \sum_{i=1}^{k}a_is_i\frac{n}{n_i}$. \textit{Thuật toán \ref{solve-crt}} giải phương trình CRT tổng quát

\begin{algorithm}
\caption{Giải hệ phương trình CRT}\label{alg:solve-crt}
\hspace*{\algorithmicindent} \textbf{Input:} listA, listN: Danh sách $a_i$ và $n_i$\\
\hspace*{\algorithmicindent} \textbf{Output:} x: Nghiệm của hệ phương trình CRT
\begin{algorithmic}[1]
\Procedure{SolveCrt}{listA, listN}
\State $len \gets \textit{sizeof(listN)}$
\State $n \gets 0$
\State $sum \gets 0$
\For {i \textbf{in} listN}
\State $n \gets n*i$
\EndFor
\For {$i \gets 1$, len}
\State $n_i \gets listN[i]$
\State $p \gets n/n_i$
\State $a_i \gets listA[i]$
\State $r,s \gets \textit{EXTEND\_EUCLID(n_i, p)}$
\State $sum \gets a_i*s*p$
\EndFor
\State \textbf{Return } $sum\ \%\ n$
\EndProcedure
\end{algorithmic}
\end{algorithm}