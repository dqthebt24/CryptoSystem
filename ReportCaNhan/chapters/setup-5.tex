Đồ án dùng định lý Fermat nhỏ để kiểm tra số số giả nguyên tố mạnh với 4 cơ sở là bốn số tự nhiên. \textit{Thuật toán \ref{alg:is-prime}} mô tả cách kiểm tra số nguyên 
tố (tham khảo giáo trình).

\begin{algorithm}
\caption{Kiểm tra số nguyên tố}\label{alg:is-prime}
\hspace*{\algorithmicindent} \textbf{Input:} BigInt n\\
\hspace*{\algorithmicindent} \textbf{Output:} True: là số nguyên tố, False: ngược lại 
\begin{algorithmic}[1]
\Procedure{IsPrime}{$n$}
\For {$i$\ \textbf{in}\ $\{2, 5, 7, 9\}$}
\If {$i^{n-1}\ \% \ n \neq 1$}
\State \textbf{Return } false
\EndIf
\EndFor
\State \textbf{Return } true
\EndProcedure
\end{algorithmic}
\end{algorithm}

\textit{Lưu ý: Có một vài số không phải là số nguyên tố nhưng thõa mãn định lý Fermat nhỏ gọi là \textbf{số Carmichael} ví dụ 3 số Carmichael nhỏ nhất là \textbf{561, 1105, 1729}. Tuy nhiên tỉ lệ gặp phải số Carmichael rất thấp\footnote{https://t.ly/1PtR} vì chỉ có khoảng \textbf{255 số Carmichael $<\ 10^8$} và khoảng \textbf{20138200 số Carmichael $<\ 10^{21}$}, do đó có thể dùng phép kiểm tra số nguyên tố như trên.}