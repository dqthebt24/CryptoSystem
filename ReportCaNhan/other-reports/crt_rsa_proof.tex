\documentclass[12pt]{article}
\usepackage[T5]{fontenc}
\usepackage[utf8]{inputenc}
\usepackage[vietnamese,english]{babel}
\usepackage{amsmath}
\usepackage{amsfonts}
\usepackage{graphicx}
\usepackage[colorinlistoftodos]{todonotes}
\usepackage{listings}
\usepackage{hyperref}
\usepackage[figurename=Hình]{caption}
\usepackage{algorithm}
\usepackage[noend]{algpseudocode}

\hypersetup{
    colorlinks=true,
    linkcolor=blue,
    filecolor=magenta,      
    urlcolor=cyan,
}

\author{Đỗ Quốc Thế}
\title{Chứng minh CRT cho RSA}
\begin{document}
\maketitle

\textbf{CRT}: Cho $n_i \in \mathcal{P},\ n_i \neq n_j, \forall i \neq j$, $a_i \in \mathbb{N}$. Hệ phương trình (\ref{sequations:abs-crt}) có nghiệm duy nhất trong $\mathbb{Z}_{n_1n_2\hdots n_k}$

\begin{equation}\label{sequations:abs-crt}
\begin{cases}
x &\equiv a_1\ (mod\ n_1) \\
x &\equiv a_2\ (mod\ n_2) \\
& \vdots \\
x &\equiv a_k\ (mod\ n_k)
\end{cases}
\end{equation}


\textbf{RSA}: Cho $p,q \in \mathcal{P},\ p \neq q,\ n\ =\ pq,\ \phi=(p-1)(q-1),\ \phi(p)=p-1,\ \phi(q)=q-1$. Chọn $e$ sao cho $gcd(e,\phi)=1$, chọn $d$ sao cho $ed\equiv\ 1\ (mod\ \phi)$.

\textbf{Encrypt}: $c=m^e\ mod\ n$

\textbf{Decrypt}: $m=c^d\ mod\ n$
\vskip 0.2in
\textbf{Ta thấy}: 

$m\ mod\ p=(c^d\ mod\ n)\ mod\ p =c^d\ mod\ p = c^{d\ mod\ \phi(p)}\ mod\ p$

$\Rightarrow m \equiv c^{d\ mod\ \phi(p)}\ (mod\ p)$


Tương tự: 

$m \equiv c^{d\ mod\ \phi(q)}\ (mod\ q)$

\vskip 0.2in
\textbf{Giải thích}: Đặt $d = k\phi(p) + d\ mod\ \phi(p)$

\begin{equation*}
\begin{array}{l@{}l}
c^d\ mod\ p &{}= c^{k\phi(p) + d\ mod\ \phi(p)}\ mod\ p\\
&{}=(c^{\phi(p)})^kc^{d\ mod\ \phi(p)}\ mod\ p \\
&{}=(1)^kc^{d\ mod\ \phi(p)}\ mod\ p \\
&{}= c^{d\ mod\ \phi(p)}\ mod\ p
\end{array}
\end{equation*}

Vậy ta có: 
\begin{equation}\label{sequations:rsa-crt}
\begin{cases}
m \equiv c^{d\ mod\ \phi(p)} (mod\ p) \\
m \equiv c^{d\ mod\ \phi(q)} (mod\ q)
\end{cases}
\end{equation}

Giải hệ CRT (2) ta được $m$

\vskip 0.2in
\textbf{Tìm nghiệm của hệ pt}:

\begin{equation}\label{sequations:abs-crt}
\begin{cases}
x &\equiv a\ (mod\ p) \\
x &\equiv b\ (mod\ q) \\
\end{cases}
\end{equation}
\begin{center}
với $p,q\in \mathcal{P}$
\end{center}

Vì $p,q\in \mathcal{P}$ $\Rightarrow \exists p'\equiv p^{-1}\ (mod\ q),\ \exists q'\equiv q^{-1}\ (mod\ p)$
\vskip 0.2in
Đặt $y=aq'q + bp'p$, ta thấy: $y \equiv a\ (mod\ p)$ và  $y \equiv b\ (mod\ q)$

$\Rightarrow y$ là nghiệm duy nhất của (3) trong $\mathbb{Z}_{pq}$
\end{document}